\section*{7. 需求的可追踪性}

{\normalsize
本测试大纲对《xxxxxxxxxx系统软件需求规格说明》的可追踪性如表14和表15所示。

% 表格使用[H]参数强制在当前位置,防止标题与内容分离
\begin{table}[H]
\centering
\vspace{6pt}
{\wuhaohei 表 14 xxxxxxxxxx与需求规格说明以及测试项的追踪关系表}

\vspace{6pt}
{\settablespacing
\begin{tabular}{|p{1cm}|p{3cm}|p{3cm}|p{7cm}|}
\hline
\rowcolor{gray!20}
\multicolumn{1}{|c|}{{\xiaowuhei 序号}} & \multicolumn{1}{c|}{{\xiaowuhei 需求名称/标识}} & \multicolumn{1}{c|}{{\xiaowuhei 需求规格说明章节号}} & \multicolumn{1}{c|}{{\xiaowuhei 测试项名称/标识}} \\
\hline
\multicolumn{1}{|c|}{{\xiaowu 1}} & {\xiaowu 数据血缘查询功能} & {\xiaowu 4.x} & {\xiaowu xxxxxxxx功能测试} \\
\hline
\multicolumn{1}{|c|}{{\xiaowu 2}} & {\xiaowu 数据可信查询功能} & {\xiaowu 4.x} & {\xiaowu xxxxxxxx功能测试} \\
\hline
\multicolumn{1}{|c|}{{\xiaowu 3}} & {\xiaowu 模块间接口} & {\xiaowu 4.x} & {\xiaowu xxxxxxxx接口测试} \\
\hline
\multicolumn{1}{|c|}{{\xiaowu 4}} & {\xiaowu 系统可靠性} & {\xiaowu 4.x} & {\xiaowu xxxxxxxx可靠性测试} \\
\hline
\end{tabular}
}
\end{table}
\vspace{6pt}  % 表格后为正文,总间距18pt

xxxxxxxxxx系统软件需求规格说明中描述的PQL查询语言交互功能,本测试大纲的第4.2节对此进行了覆盖。

% 表格使用[H]参数强制在当前位置,防止标题与内容分离
\begin{table}[H]
\centering
\vspace{6pt}
{\wuhaohei 表 15 xxxxxxxxxx与需求规格说明以及测试项的逆向追踪关系表}

\vspace{6pt}
{\settablespacing
\begin{tabular}{|p{1cm}|p{3cm}|p{3cm}|p{7cm}|}
\hline
\rowcolor{gray!20}
\multicolumn{1}{|c|}{{\xiaowuhei 序号}} & \multicolumn{1}{c|}{{\xiaowuhei 测试项名称/标识}} & \multicolumn{1}{c|}{{\xiaowuhei 本文档的章节号}} & \multicolumn{1}{c|}{{\xiaowuhei 需求名称/标识}} \\
\hline
\multicolumn{1}{|c|}{{\xiaowu 1}} & {\xiaowu xxxxxxxx功能测试} & {\xiaowu 4.1、4.2} & {\xiaowu 数据血缘查询功能、数据可信查询功能} \\
\hline
\multicolumn{1}{|c|}{{\xiaowu 2}} & {\xiaowu xxxxxxxx接口测试} & {\xiaowu 4.2} & {\xiaowu 模块间接口} \\
\hline
\multicolumn{1}{|c|}{{\xiaowu 3}} & {\xiaowu xxxxxxxx可靠性测试} & {\xiaowu 4.2} & {\xiaowu 系统可靠性} \\
\hline
\end{tabular}
}
\end{table}
\vspace{-6pt}  % 表格后为标题,总间距6pt

}

% =================== 第8章:注释 ===================
\noindent