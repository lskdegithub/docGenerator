\section*{1. 范围}

{\normalsize
\subsection*{1.1 标识}

{\normalsize
(1)文档标识号:xxxx;

(2)系统标识:xxxx,测试文档简称为xxxx;

(3)项目名称:xxxxxx;

(4)文档名称:xxxxxx节点软件测试大纲;

(5)软件版本号:xxxx版,软件版本标识采用三位编码规则,项目文档版本标识采用四位编码规则;

(6)本文档适用于:xxxxx系统项目。
}

\subsection*{1.2 系统概述}

{\normalsize
系统用途部分内容敏感,不在此处列出。详见xxxxxx系统合同(xxxxxx)。

xxxxxxx规定的本阶段主要要求和技术指标为:

xxxxxxxxxxxxxxxxxxxxxxxxxxxx。

项目节点测试的测试项与xxxxxxx的覆盖性对应关系如表1所示。

% 表格使用[H]参数强制在当前位置,防止标题与内容分离
\begin{table}[H]
\centering
\vspace{6pt}
{\wuhaohei 表 1 xxxxxx与测试项覆盖性对照表}

\vspace{6pt}
{\settablespacing
\begin{tabular}{|p{0.5cm}|p{7cm}|p{7cm}|}
\hline
\rowcolor{gray!20}
\multicolumn{1}{|c|}{{\xiaowuhei 序号}} & \multicolumn{1}{c|}{{\xiaowuhei xxxxxx}} & \multicolumn{1}{c|}{{\xiaowuhei 测试项}} \\
\hline
\multicolumn{1}{|c|}{{\xiaowu 1}} & {\xiaowu } & {\xiaowu } \\
\hline
\multicolumn{1}{|c|}{{\xiaowu 2}} & {\xiaowu } & {\xiaowu } \\
\hline
\multicolumn{1}{|c|}{{\xiaowu 3}} & {\xiaowu } & {\xiaowu } \\
\hline
\multicolumn{1}{|c|}{{\xiaowu 4}} & {\xiaowu } & {\xiaowu } \\
\hline
\multicolumn{1}{|c|}{{\xiaowu 5}} & {\xiaowu } & {\xiaowu } \\
\hline
\end{tabular}
}
\end{table}
\vspace{6pt}  % 表格后为正文,总间距18pt

xxxxxxxxxx系统软件的需方是"M"项目管理办公室,开发方是xxxxx。

标识当前和计划运行的现场,测试地点为"xxxxxx",测试环境包括1台国产ARM架构服务器,其搭载国产处理器(飞腾)和国产操作系统(银河麒麟服务器版)以及2台国产X86台式机,其中服务器部署xxxxxxxxxxreleaseV0.7.1版,两台台式机为客户端。

\subsection*{1.3 文档概述}

{\normalsize
本文档用于"xxxxxxxxxx系统"项目2025年9月节点测试大纲,规定了本次测试的类型、内容、环境、方法、条件及各项测试活动的进度安排,指导测试人员进行测试活动,为编写软件测试细则以及测试报告提供依据及为后续测试工作打下基础。

本文档模板涵盖了GJB438C-2021《军用软件开发文档通用要求》对软件测试大纲文档的要素和内容的要求。

描述本文档密级为xx密,必须按照机密文件的要求使用和分发。即使在保密性得到保证的情况下,对本文档的使用也必须经研制方主管人员的认可。
}

\subsection*{1.4 与其他计划的关系}

{\normalsize
本文档属于"xxxxxxxxxx系统"项目2025年9月节点测试的一部分,测试人员根据xxxxxxxxxx系统项目《软件需求规格说明》识别被测软件的功能和结构,确定测试环境,收集测试需求以标识被测试项,安排测试进度以及进行测试活动。
}

}

% =================== 第2章:引用文档 ===================
\noindent