\section*{6. 测试终止条件}

{\normalsize
本次测试的终止条件如表13所示。

% 表格使用[H]参数强制在当前位置,防止标题与内容分离
\begin{table}[H]
\centering
\vspace{6pt}
{\wuhaohei 表 13 缺陷级别}

\vspace{6pt}
{\settablespacing
\begin{tabular}{|p{1.5cm}|p{3cm}|p{10cm}|}
\hline
\rowcolor{gray!20}
\multicolumn{1}{|c|}{{\xiaowuhei 缺陷级别}} & \multicolumn{1}{c|}{{\xiaowuhei 描述}} & \multicolumn{1}{c|}{{\xiaowuhei 说明}} \\
\hline
\multicolumn{1}{|c|}{{\xiaowu 1级}} & {\xiaowu 致命问题} & {\xiaowu 程序运行过程中不断申请但没有完全释放资源,造成系统性能越来越低并出现无规律的死机现象。程序运行导致系统崩溃。由程序引起的资源严重不足、非法退出等。严重的关键计算错误(如计费等)。数据库发生死锁且无法自动恢复。与需求要求差距较大。系统无响应} \\
\hline
\multicolumn{1}{|c|}{{\xiaowu 2级}} & {\xiaowu 严重问题} & {\xiaowu 因错误操作导致的程序中断。功能没有实现。正确操作导致的错误结果。与数据库连接错误且无法自动恢复。数据通讯错误且无法自动恢复。数据库的表、业务规则、缺省值未加完整性等约束条件。界面中的信息不能及时刷新不能正确反映当前数据状态可能误导用户(数据库中剩余记录个数和参数设置对话框中的预设值常常显示为历史值而不是当前值)。对输入数据没有进行充分并且有效的有效性检查造成不合要求的数据进入数据库} \\
\hline
\multicolumn{1}{|c|}{{\xiaowu 3级}} & {\xiaowu 一般问题} & {\xiaowu 操作界面错误(包括数据窗口内列名定义、含义是否一致界面中英文混杂界面元素参差不齐文字显示不全)。打印内容、格式错误。删除操作未给出提示。数据库表中有过多的空字段。提示信息意文不明或为原始的英文提示。要求用户输入多余的、本来系统可以自动获取的数据(服务是否启动安装后用户需要手动修改某些配置文件)。辅助说明描述不清楚有歧义。长操作未给用户提示} \\
\hline
\multicolumn{1}{|c|}{{\xiaowu 4级}} & {\xiaowu 改进建议} & {\xiaowu 辅助说明描述不清楚。输入输出不规范。可输入区域和只读区域没有明显的区分标志。某一项功能的冗余操作太多如对话框嵌套层次太多影响用户操作。不能记忆用户的设置或操作习惯用户每次进入系统都需要重新操作初始环境。不符合用户操作习惯(快捷键定义不科学不实用键位分布不合理按键太多甚至没有快捷键)。界面不规范。提示窗口文字未采用行业术语} \\
\hline
\end{tabular}
}
\end{table}
\vspace{6pt}  % 表格后为正文,总间距18pt

测试终止条件为:
(1)全部测试用例被执行完毕;
(2)测试问题报告中1级、2级问题全部关闭;
(3)测试问题报告中3级、4级问题关闭率达到80\%以上。
}

% =================== 第7章:需求的可追踪性 ===================
\noindent