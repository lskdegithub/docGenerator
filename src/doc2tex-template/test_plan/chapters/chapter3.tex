\section*{3. 软件测试环境}

{\normalsize
\subsection*{3.1 测试现场}

{\normalsize
xxxxxxxxxxxxxxxxxxxxxxxxxxxxxxxxxxxxxxxxxxxxxxxxxxxxxxxxxxxxxxxxxxxxxxxxxxxxxx。

测试现场位于xxxxxx,符合测试环境要求,具备必要的测试条件和安全保障措施。
}

\subsection*{3.2 测试环境概述}

{\normalsize
xxxxxxxxxxxxxxxxxxxxxxxxxxxxxxxxxxxxxxxxxxxxxxxxxxxxxxxxxxxxxxxxxxxxxxxxxxxxxx。

测试环境总体架构如下:

(1)硬件环境:包括服务器、客户端、网络设备等;

(2)软件环境:包括操作系统、数据库系统、中间件等;

(3)网络环境:xxx局域网,带宽xxxMbps;

(4)测试工具:xxxxxxxxxxxxxxxxxxxxxxxxxxxxxxxxxxxx。
}

\subsubsection*{3.2.1 软件项}

{\normalsize
测试环境中的软件项如表3所示。

% 表格使用[H]参数强制在当前位置,防止标题与内容分离
\begin{table}[H]
\centering
\vspace{6pt}
{\wuhaohei 表 3 测试环境被测软件}

\vspace{6pt}
{\settablespacing
\begin{tabular}{|p{0.5cm}|p{3.5cm}|p{3cm}|p{3cm}|p{2.5cm}|p{2cm}|}
\hline
\rowcolor{gray!20}
\multicolumn{1}{|c|}{{\xiaowuhei 序号}} & \multicolumn{1}{c|}{{\xiaowuhei 软件名称}} & \multicolumn{1}{c|}{{\xiaowuhei 版本号}} & \multicolumn{1}{c|}{{\xiaowuhei 用途}} & \multicolumn{1}{c|}{{\xiaowuhei 供应商}} & \multicolumn{1}{c|}{{\xiaowuhei 备注}} \\
\hline
\multicolumn{1}{|c|}{{\xiaowu 1}} & {\xiaowu xxxxxxxx系统} & {\xiaowu V0.7.1} & {\xiaowu 被测软件} & {\xiaowu xxxxx} & {\xiaowu } \\
\hline
\multicolumn{1}{|c|}{{\xiaowu 2}} & {\xiaowu 银河麒麟服务器版} & {\xiaowu V10} & {\xiaowu 操作系统} & {\xiaowu 麒麟软件} & {\xiaowu 国产} \\
\hline
\multicolumn{1}{|c|}{{\xiaowu 3}} & {\xiaowu 数据库系统} & {\xiaowu V5.0} & {\xiaowu 数据存储} & {\xiaowu xxxxx} & {\xiaowu } \\
\hline
\end{tabular}
}
\end{table}
\vspace{6pt}  % 表格后为正文,总间距18pt

测试环境中的支撑软件包括操作系统、数据库系统等,为被测软件提供必要的运行环境。
}

\subsubsection*{3.2.2 硬件和固件项}

{\normalsize
测试环境中的硬件和固件项如表4所示。

% 表格使用[H]参数强制在当前位置,防止标题与内容分离
\begin{table}[H]
\centering
\vspace{6pt}
{\wuhaohei 表 5 测试环境硬件项}

\vspace{6pt}
{\settablespacing
\begin{tabular}{|p{0.5cm}|p{3.5cm}|p{3.5cm}|p{1.5cm}|p{3.5cm}|p{2cm}|}
\hline
\rowcolor{gray!20}
\multicolumn{1}{|c|}{{\xiaowuhei 序号}} & \multicolumn{1}{c|}{{\xiaowuhei 设备名称}} & \multicolumn{1}{c|}{{\xiaowuhei 型号/规格}} & \multicolumn{1}{c|}{{\xiaowuhei 数量}} & \multicolumn{1}{c|}{{\xiaowuhei 性能参数}} & \multicolumn{1}{c|}{{\xiaowuhei 备注}} \\
\hline
\multicolumn{1}{|c|}{{\xiaowu 1}} & {\xiaowu 服务器} & {\xiaowu 飞腾FT-2000+} & {\xiaowu 1台} & {\xiaowu 64核,128GB内存} & {\xiaowu 国产} \\
\hline
\multicolumn{1}{|c|}{{\xiaowu 2}} & {\xiaowu 台式机} & {\xiaowu 联想开天N8系列} & {\xiaowu 2台} & {\xiaowu Intel i7,16GB内存} & {\xiaowu 客户端} \\
\hline
\multicolumn{1}{|c|}{{\xiaowu 3}} & {\xiaowu 网络设备} & {\xiaowu xxxxx} & {\xiaowu 1台} & {\xiaowu 千兆以太网} & {\xiaowu } \\
\hline
\end{tabular}
}
\end{table}
\vspace{6pt}  % 表格后为正文,总间距18pt

硬件配置满足测试要求,所有设备均已安装调试完毕,运行正常。固件版本符合系统要求。
}

\subsubsection*{3.2.3 其他材料}

{\normalsize
测试所需的其他材料包括:

(1)测试数据:xxxxxxxxxxxxxxxxxxxxxxxxxxxxxxxxxxxx;

(2)测试用例:xxxxxxxxxxxxxxxxxxxxxxxxxxxxxxxxxxxx;

(3)用户手册:xxxxxxxxxxxxxxxxxxxxxxxxxxxxxxxxxxxx;

(4)技术文档:xxxxxxxxxxxxxxxxxxxxxxxxxxxxxxxxxxxx。
}

\subsubsection*{3.2.4 所有者的特性、需方权利和许可证}

{\normalsize
xxxxxxxxxxxxxxxxxxxxxxxxxxxxxxxxxxxxxxxxxxxxxxxxxxxxxxxxxxxxxxxxxxxxxxxxxxxxxx。

(1)软件所有权:xxxxxxxxxxxxxxxxxxxxxxxxxxxxxxxxxxxx;

(2)需方权利:xxxxxxxxxxxxxxxxxxxxxxxxxxxxxxxxxxxx;

(3)许可证要求:xxxxxxxxxxxxxxxxxxxxxxxxxxxxxxxxxxxx。
}

\subsubsection*{3.2.5 安装、测试和控制}

{\normalsize
在执行测试前,要求完成以下准备工作:

(1)根据测试大纲,组织测试组,安排具体的测试人员及其分工;

(2)获取测试大纲中拟定的所需要的所有软件、硬件、数据、文档;

(3)测试环境的每一项在使用前必须进行安装和测试;

(4)测试环境的每一项必须可控制,并进行相应的维护。

根据测试大纲获取和准备测试环境中的每个元素。在测试开始前,从受控库提取被测软件,配置测试环境。测试环境的准备、建立与维护计划表如表6所示。

% 表格使用[H]参数强制在当前位置,防止标题与内容分离
\begin{table}[H]
\centering
\vspace{6pt}
{\wuhaohei 表 6 测试环境的准备、建立与维护计划表}
\nobreak
\vspace{6pt}
{\settablespacing
\begin{tabular}{|p{1.5cm}|p{6.5cm}|p{3.5cm}|p{3cm}|}
\hline
\rowcolor{gray!20}
\multicolumn{1}{|c|}{{\xiaowuhei 序号}} & \multicolumn{1}{c|}{{\xiaowuhei 工作}} & \multicolumn{1}{c|}{{\xiaowuhei 职责}} & \multicolumn{1}{c|}{{\xiaowuhei 负责人}} \\
\hline
\multicolumn{1}{|c|}{{\xiaowu 1}} & {\xiaowu 获取和开发测试环境元素} & {\xiaowu 逐项获取或开发软件测试环境涉及到的软件项、硬件和固件项、其他项、其他材料等} & {\xiaowu 李胜奎} \\
\hline
\multicolumn{1}{|c|}{{\xiaowu 2}} & {\xiaowu 建立软件测试环境} & {\xiaowu 测试开始前,安装调试软件测试环境,使其达到测试要求} & {\xiaowu 杭滔} \\
\hline
\multicolumn{1}{|c|}{{\xiaowu 3}} & {\xiaowu 控制和维护软件测试环境} & {\xiaowu 在测试过程中,控制和维护软件测试环境中的每个项} & {\xiaowu 李胜奎} \\
\hline
\end{tabular}
}
\end{table}
\vspace{-6pt}  % 表格后为标题,总间距6pt

}

\subsubsection*{3.2.6 测试环境的差异性分析和有效性说明}

{\normalsize
xxxxxxxxxx测试环境满足合同指标要求中对计算机资源的要求,对测试结果不产生影响。
}

\subsubsection*{3.2.7 参与组织}

{\normalsize
本次测试的参与组织如表7所示。

% 表格使用[H]参数强制在当前位置,防止标题与内容分离
\begin{table}[H]
\centering
\vspace{6pt}
{\wuhaohei 表 7 参与组织}

\vspace{6pt}
{\settablespacing
\begin{tabular}{|p{1cm}|p{3.5cm}|p{6.5cm}|p{3.5cm}|}
\hline
\rowcolor{gray!20}
\multicolumn{1}{|c|}{{\xiaowuhei 序号}} & \multicolumn{1}{c|}{{\xiaowuhei 参与组织}} & \multicolumn{1}{c|}{{\xiaowuhei 职责}} & \multicolumn{1}{c|}{{\xiaowuhei 负责人}} \\
\hline
\multicolumn{1}{|c|}{{\xiaowu 1}} & {\xiaowu 测试组} & {\xiaowu 策划测试活动、测试用例设计及实现、组织实施测试、测试总结、出具测试报告} & {\xiaowu 李胜奎} \\
\hline
\multicolumn{1}{|c|}{{\xiaowu 2}} & {\xiaowu 开发组} & {\xiaowu 参与测试需求分析,明确测试的范围和边界,解决测试过程中发现的bug} & {\xiaowu 陈幻} \\
\hline
\multicolumn{1}{|c|}{{\xiaowu 3}} & {\xiaowu 质量管理组} & {\xiaowu 测试过程中的质量管理相关工作} & {\xiaowu 任静、李闺臣} \\
\hline
\multicolumn{1}{|c|}{{\xiaowu 4}} & {\xiaowu 配置管理组} & {\xiaowu 出具测试版本,测试过程中的版本控制} & {\xiaowu 杭滔} \\
\hline
\end{tabular}
}
\end{table}
\vspace{-6pt}  % 表格后为标题,总间距6pt

}

\subsubsection*{3.2.8 人员与分工}

{\normalsize
本次测试人员与分工安排如表8所示。

% 表格使用[H]参数强制在当前位置,防止标题与内容分离
\begin{table}[H]
\centering
\vspace{6pt}
{\wuhaohei 表 8 人员与分工安排}

\vspace{6pt}
{\settablespacing
\begin{tabular}{|p{0.5cm}|p{4.5cm}|p{2.5cm}|p{2.5cm}|p{4.5cm}|}
\hline
\rowcolor{gray!20}
\multicolumn{1}{|c|}{{\xiaowuhei 序号}} & \multicolumn{1}{c|}{{\xiaowuhei 测试活动名称}} & \multicolumn{1}{c|}{{\xiaowuhei 开始日期}} & \multicolumn{1}{c|}{{\xiaowuhei 结束日期}} & \multicolumn{1}{c|}{{\xiaowuhei 人员}} \\
\hline
\multicolumn{1}{|c|}{{\xiaowu 1}} & {\xiaowu 组织测试策划、测试需求分析,编写测试大纲} & {\xiaowu 2025.9.10} & {\xiaowu 2025.9.15} & {\xiaowu 李胜奎,李闺臣} \\
\hline
\multicolumn{1}{|c|}{{\xiaowu 2}} & {\xiaowu 测试设计,设计测试用例,编写测试细则、部署测试环境} & {\xiaowu 2025.9.13} & {\xiaowu 2025.9.18} & {\xiaowu 喻昕、胡栋、李胜奎、杭滔} \\
\hline
\multicolumn{1}{|c|}{{\xiaowu 3}} & {\xiaowu 开展功能测试(包含回归测试)并记录测试结果} & {\xiaowu 2025.9.18} & {\xiaowu 2025.9.25} & {\xiaowu 喻昕、胡栋} \\
\hline
\multicolumn{1}{|c|}{{\xiaowu 4}} & {\xiaowu 负责监督软件测试过程、软件问题报告单的闭环控制、评估测试活动} & {\xiaowu 2025.9.18} & {\xiaowu 2025.9.25} & {\xiaowu 周文俊,钟佳惠} \\
\hline
\multicolumn{1}{|c|}{{\xiaowu 5}} & {\xiaowu 负责软件版本管理,测试文档的维护管理} & {\xiaowu 2025.9.10} & {\xiaowu 2025.9.25} & {\xiaowu 李闺臣、李胜奎} \\
\hline
\multicolumn{1}{|c|}{{\xiaowu 6}} & {\xiaowu 测试总结,分析测试数据,总结测试工作,编写测试报告} & {\xiaowu 2025.9.20} & {\xiaowu 2025.9.30} & {\xiaowu 李胜奎、喻昕} \\
\hline
\end{tabular}
}
\end{table}
\vspace{-6pt}  % 表格后为标题,总间距6pt

}

\subsubsection*{3.2.9 人员培训}

{\normalsize
本次测试相关人员均有相关测试经验,故无需进行人员培训。
}

\subsubsection*{3.2.10 要执行的测试}

{\normalsize
要执行的测试详见4.2节。
}

}

% =================== 第4章:测试说明 ===================
\noindent