% 技术文件 - 严格按照格式标准
\documentclass[12pt,a4paper]{article}

% 使用标准包
\usepackage{ctex}
\usepackage{fontspec}  % 字体设置
\usepackage{xcolor}  % 用于行颜色
\usepackage{colortbl}  % 表格颜色支持
\usepackage{geometry}  % 页面设置
\usepackage{setspace}  % 行间距设置
\usepackage{titlesec}  % 标题格式设置
\usepackage{float}    % 浮动体控制,用于[H]参数

% 按标准设置中文字体
\setCJKmainfont{Noto Serif CJK SC}  % 宋体
\setCJKsansfont{Noto Sans CJK SC}  % 黑体
\setmainfont{DejaVu Serif}  % 西文字体

% 按标准定义字号命令
\newcommand{\wuhao}{\fontsize{10.5pt}{18pt}\selectfont}  % 五号,行间距18磅
\newcommand{\wuhaohei}{\wuhao\bfseries}  % 五号黑体加粗
\newcommand{\xiaowu}{\fontsize{9pt}{11pt}\selectfont}  % 小五号
\newcommand{\xiaowusong}{\xiaowu\rmfamily}  % 小五号宋体
\newcommand{\xiaowuhei}{\xiaowu\bfseries}  % 小五号黑体加粗
\newcommand{\sanhao}{\fontsize{16pt}{18pt}\selectfont\bfseries}  % 三号黑体加粗

% 按标准定义正文字体为五号宋体,西文为Times New Roman
\renewcommand{\normalsize}{\wuhao}

% 页面设置
\geometry{a4paper,left=2.5cm,right=2.5cm,top=2.5cm,bottom=2.5cm}

% 设置段落首行缩进为标准中文缩进
\setlength{\parindent}{1.5em}
\setlength{\parskip}{0pt}

% 设置全局行间距倍数为1.0(已在\fontsize中设置行间距为18pt)
\renewcommand{\baselinestretch}{1.0}

% 设置页码为小五号新罗马体,版心下面居中
\pagestyle{plain}
\renewcommand{\thepage}{\xiaowu\rmfamily\arabic{page}}

% 表格标题格式 - 五号黑体
\renewcommand{\tablename}{\wuhaohei 表}
\renewcommand{\figurename}{\wuhaohei 图}

% === 表格行间距设置 ===
% 定义表格行间距倍数(设置为1.0,使用默认表格行间距)
\newcommand{\tablearraystretch}{1.0}
% 应用表格行间距的辅助命令
\newcommand{\settablespacing}{\renewcommand{\arraystretch}{\tablearraystretch}}
\newcommand{\resettablespacing}{\renewcommand{\arraystretch}{1.0}}

% 设置1级、2级、3级标题格式:五号宋体加粗,标题之间间距6磅,行间距18磅
\titleformat{\section}
  {\wuhao\bfseries}
  {\thesection}
  {1em}
  {}
\titlespacing*{\section}{0pt}{6pt}{0pt}  % 段前6pt,段后0pt,标题之间间距为6pt

\titleformat{\subsection}
  {\wuhao\bfseries}
  {\thesubsection}
  {1em}
  {}
\titlespacing*{\subsection}{0pt}{6pt}{0pt}  % 段前6pt,段后0pt,标题之间间距为6pt

\titleformat{\subsubsection}
  {\wuhao\bfseries}
  {\thesubsubsection}
  {1em}
  {}
\titlespacing*{\subsubsection}{0pt}{6pt}{0pt}  % 段前6pt,段后0pt,标题之间间距为6pt

% 四级标题设置
\titleformat{\paragraph}
  {\wuhao\bfseries}
  {}
  {0em}
  {}
\titlespacing*{\paragraph}{0pt}{6pt}{0pt}  % 段前6pt,段后0pt,与其他标题保持一致

% 五级标题设置
\titleformat{\subparagraph}
  {\wuhao\bfseries}
  {}
  {0em}
  {}
\titlespacing*{\subparagraph}{0pt}{6pt}{0pt}  % 段前6pt,段后0pt,与其他标题保持一致

\begin{document}

% 移除封面和目录,直接从第1章开始

% =================== 第一章:范围 ===================
\noindent\section*{1. 范围}

{\normalsize
\subsection*{1.1 标识}

{\normalsize
(1)文档标识号:xxxx;

(2)系统标识:xxxx,测试文档简称为xxxx;

(3)项目名称:xxxxxx;

(4)文档名称:xxxxxx节点软件测试大纲;

(5)软件版本号:xxxx版,软件版本标识采用三位编码规则,项目文档版本标识采用四位编码规则;

(6)本文档适用于:xxxxx系统项目。
}

\subsection*{1.2 系统概述}

{\normalsize
系统用途部分内容敏感,不在此处列出。详见xxxxxx系统合同(xxxxxx)。

xxxxxxx规定的本阶段主要要求和技术指标为:

xxxxxxxxxxxxxxxxxxxxxxxxxxxx。

项目节点测试的测试项与xxxxxxx的覆盖性对应关系如表1所示。

% 表格使用[H]参数强制在当前位置,防止标题与内容分离
\begin{table}[H]
\centering
\vspace{6pt}
{\wuhaohei 表 1 xxxxxx与测试项覆盖性对照表}

\vspace{6pt}
{\settablespacing
\begin{tabular}{|p{0.5cm}|p{7cm}|p{7cm}|}
\hline
\rowcolor{gray!20}
\multicolumn{1}{|c|}{{\xiaowuhei 序号}} & \multicolumn{1}{c|}{{\xiaowuhei xxxxxx}} & \multicolumn{1}{c|}{{\xiaowuhei 测试项}} \\
\hline
\multicolumn{1}{|c|}{{\xiaowu 1}} & {\xiaowu } & {\xiaowu } \\
\hline
\multicolumn{1}{|c|}{{\xiaowu 2}} & {\xiaowu } & {\xiaowu } \\
\hline
\multicolumn{1}{|c|}{{\xiaowu 3}} & {\xiaowu } & {\xiaowu } \\
\hline
\multicolumn{1}{|c|}{{\xiaowu 4}} & {\xiaowu } & {\xiaowu } \\
\hline
\multicolumn{1}{|c|}{{\xiaowu 5}} & {\xiaowu } & {\xiaowu } \\
\hline
\end{tabular}
}
\end{table}
\vspace{6pt}  % 表格后为正文,总间距18pt

xxxxxxxxxx系统软件的需方是"M"项目管理办公室,开发方是xxxxx。

标识当前和计划运行的现场,测试地点为"xxxxxx",测试环境包括1台国产ARM架构服务器,其搭载国产处理器(飞腾)和国产操作系统(银河麒麟服务器版)以及2台国产X86台式机,其中服务器部署xxxxxxxxxxreleaseV0.7.1版,两台台式机为客户端。

\subsection*{1.3 文档概述}

{\normalsize
本文档用于"xxxxxxxxxx系统"项目2025年9月节点测试大纲,规定了本次测试的类型、内容、环境、方法、条件及各项测试活动的进度安排,指导测试人员进行测试活动,为编写软件测试细则以及测试报告提供依据及为后续测试工作打下基础。

本文档模板涵盖了GJB438C-2021《军用软件开发文档通用要求》对软件测试大纲文档的要素和内容的要求。

描述本文档密级为xx密,必须按照机密文件的要求使用和分发。即使在保密性得到保证的情况下,对本文档的使用也必须经研制方主管人员的认可。
}

\subsection*{1.4 与其他计划的关系}

{\normalsize
本文档属于"xxxxxxxxxx系统"项目2025年9月节点测试的一部分,测试人员根据xxxxxxxxxx系统项目《软件需求规格说明》识别被测软件的功能和结构,确定测试环境,收集测试需求以标识被测试项,安排测试进度以及进行测试活动。
}

}

% =================== 第2章:引用文档 ===================
\noindent\section*{2. 引用文档}

{\normalsize
本文档引用的文档一览表如表2所示。

% 表格使用[H]参数强制在当前位置,防止标题与内容分离
\begin{table}[H]
\centering
\vspace{6pt}
{\wuhaohei 表 2 引用文档一览表}

\vspace{6pt}
{\settablespacing
\begin{tabular}{|p{0.5cm}|p{2.5cm}|p{5.5cm}|p{2.5cm}|p{1.5cm}|p{2cm}|}
\hline
\rowcolor{gray!20}
\multicolumn{1}{|c|}{{\xiaowuhei 序号}} & \multicolumn{1}{c|}{{\xiaowuhei 编号}} & \multicolumn{1}{c|}{{\xiaowuhei 标题}} & \multicolumn{1}{c|}{{\xiaowuhei 编写单位}} & \multicolumn{1}{c|}{{\xiaowuhei 日期}} & \multicolumn{1}{c|}{{\xiaowuhei 备注}} \\
\hline
\multicolumn{1}{|c|}{{\xiaowu 1}} & {\xiaowu GJB 438C-2021} & {\xiaowu 军用软件开发文档通用要求} & {\xiaowu xxxxx} & {\xiaowu 2021} & {\xiaowu /} \\
\hline
\multicolumn{1}{|c|}{{\xiaowu 2}} & {\xiaowu /} & {\xiaowu xxxxxxxxxx} & {\xiaowu xxxxx} & {\xiaowu 2025} & {\xiaowu /} \\
\hline
\multicolumn{1}{|c|}{{\xiaowu 3}} & {\xiaowu xxxxxxxxxx–SRS} & {\xiaowu xxxxxxxxxx系统软件需求规格说明} & {\xiaowu xxxxx} & {\xiaowu 2025} & {\xiaowu /} \\
\hline
\end{tabular}
}
\end{table}
\vspace{-6pt}  % 表格后为标题,总间距6pt

}

% =================== 第3章:软件测试环境 ===================
\noindent\section*{3. 软件测试环境}

{\normalsize
\subsection*{3.1 测试现场}

{\normalsize
xxxxxxxxxxxxxxxxxxxxxxxxxxxxxxxxxxxxxxxxxxxxxxxxxxxxxxxxxxxxxxxxxxxxxxxxxxxxxx。

测试现场位于xxxxxx,符合测试环境要求,具备必要的测试条件和安全保障措施。
}

\subsection*{3.2 测试环境概述}

{\normalsize
xxxxxxxxxxxxxxxxxxxxxxxxxxxxxxxxxxxxxxxxxxxxxxxxxxxxxxxxxxxxxxxxxxxxxxxxxxxxxx。

测试环境总体架构如下:

(1)硬件环境:包括服务器、客户端、网络设备等;

(2)软件环境:包括操作系统、数据库系统、中间件等;

(3)网络环境:xxx局域网,带宽xxxMbps;

(4)测试工具:xxxxxxxxxxxxxxxxxxxxxxxxxxxxxxxxxxxx。
}

\subsubsection*{3.2.1 软件项}

{\normalsize
测试环境中的软件项如表3所示。

% 表格使用[H]参数强制在当前位置,防止标题与内容分离
\begin{table}[H]
\centering
\vspace{6pt}
{\wuhaohei 表 3 测试环境被测软件}

\vspace{6pt}
{\settablespacing
\begin{tabular}{|p{0.5cm}|p{3.5cm}|p{3cm}|p{3cm}|p{2.5cm}|p{2cm}|}
\hline
\rowcolor{gray!20}
\multicolumn{1}{|c|}{{\xiaowuhei 序号}} & \multicolumn{1}{c|}{{\xiaowuhei 软件名称}} & \multicolumn{1}{c|}{{\xiaowuhei 版本号}} & \multicolumn{1}{c|}{{\xiaowuhei 用途}} & \multicolumn{1}{c|}{{\xiaowuhei 供应商}} & \multicolumn{1}{c|}{{\xiaowuhei 备注}} \\
\hline
\multicolumn{1}{|c|}{{\xiaowu 1}} & {\xiaowu xxxxxxxx系统} & {\xiaowu V0.7.1} & {\xiaowu 被测软件} & {\xiaowu xxxxx} & {\xiaowu } \\
\hline
\multicolumn{1}{|c|}{{\xiaowu 2}} & {\xiaowu 银河麒麟服务器版} & {\xiaowu V10} & {\xiaowu 操作系统} & {\xiaowu 麒麟软件} & {\xiaowu 国产} \\
\hline
\multicolumn{1}{|c|}{{\xiaowu 3}} & {\xiaowu 数据库系统} & {\xiaowu V5.0} & {\xiaowu 数据存储} & {\xiaowu xxxxx} & {\xiaowu } \\
\hline
\end{tabular}
}
\end{table}
\vspace{6pt}  % 表格后为正文,总间距18pt

测试环境中的支撑软件包括操作系统、数据库系统等,为被测软件提供必要的运行环境。
}

\subsubsection*{3.2.2 硬件和固件项}

{\normalsize
测试环境中的硬件和固件项如表4所示。

% 表格使用[H]参数强制在当前位置,防止标题与内容分离
\begin{table}[H]
\centering
\vspace{6pt}
{\wuhaohei 表 5 测试环境硬件项}

\vspace{6pt}
{\settablespacing
\begin{tabular}{|p{0.5cm}|p{3.5cm}|p{3.5cm}|p{1.5cm}|p{3.5cm}|p{2cm}|}
\hline
\rowcolor{gray!20}
\multicolumn{1}{|c|}{{\xiaowuhei 序号}} & \multicolumn{1}{c|}{{\xiaowuhei 设备名称}} & \multicolumn{1}{c|}{{\xiaowuhei 型号/规格}} & \multicolumn{1}{c|}{{\xiaowuhei 数量}} & \multicolumn{1}{c|}{{\xiaowuhei 性能参数}} & \multicolumn{1}{c|}{{\xiaowuhei 备注}} \\
\hline
\multicolumn{1}{|c|}{{\xiaowu 1}} & {\xiaowu 服务器} & {\xiaowu 飞腾FT-2000+} & {\xiaowu 1台} & {\xiaowu 64核,128GB内存} & {\xiaowu 国产} \\
\hline
\multicolumn{1}{|c|}{{\xiaowu 2}} & {\xiaowu 台式机} & {\xiaowu 联想开天N8系列} & {\xiaowu 2台} & {\xiaowu Intel i7,16GB内存} & {\xiaowu 客户端} \\
\hline
\multicolumn{1}{|c|}{{\xiaowu 3}} & {\xiaowu 网络设备} & {\xiaowu xxxxx} & {\xiaowu 1台} & {\xiaowu 千兆以太网} & {\xiaowu } \\
\hline
\end{tabular}
}
\end{table}
\vspace{6pt}  % 表格后为正文,总间距18pt

硬件配置满足测试要求,所有设备均已安装调试完毕,运行正常。固件版本符合系统要求。
}

\subsubsection*{3.2.3 其他材料}

{\normalsize
测试所需的其他材料包括:

(1)测试数据:xxxxxxxxxxxxxxxxxxxxxxxxxxxxxxxxxxxx;

(2)测试用例:xxxxxxxxxxxxxxxxxxxxxxxxxxxxxxxxxxxx;

(3)用户手册:xxxxxxxxxxxxxxxxxxxxxxxxxxxxxxxxxxxx;

(4)技术文档:xxxxxxxxxxxxxxxxxxxxxxxxxxxxxxxxxxxx。
}

\subsubsection*{3.2.4 所有者的特性、需方权利和许可证}

{\normalsize
xxxxxxxxxxxxxxxxxxxxxxxxxxxxxxxxxxxxxxxxxxxxxxxxxxxxxxxxxxxxxxxxxxxxxxxxxxxxxx。

(1)软件所有权:xxxxxxxxxxxxxxxxxxxxxxxxxxxxxxxxxxxx;

(2)需方权利:xxxxxxxxxxxxxxxxxxxxxxxxxxxxxxxxxxxx;

(3)许可证要求:xxxxxxxxxxxxxxxxxxxxxxxxxxxxxxxxxxxx。
}

\subsubsection*{3.2.5 安装、测试和控制}

{\normalsize
在执行测试前,要求完成以下准备工作:

(1)根据测试大纲,组织测试组,安排具体的测试人员及其分工;

(2)获取测试大纲中拟定的所需要的所有软件、硬件、数据、文档;

(3)测试环境的每一项在使用前必须进行安装和测试;

(4)测试环境的每一项必须可控制,并进行相应的维护。

根据测试大纲获取和准备测试环境中的每个元素。在测试开始前,从受控库提取被测软件,配置测试环境。测试环境的准备、建立与维护计划表如表6所示。

% 表格使用[H]参数强制在当前位置,防止标题与内容分离
\begin{table}[H]
\centering
\vspace{6pt}
{\wuhaohei 表 6 测试环境的准备、建立与维护计划表}
\nobreak
\vspace{6pt}
{\settablespacing
\begin{tabular}{|p{1.5cm}|p{6.5cm}|p{3.5cm}|p{3cm}|}
\hline
\rowcolor{gray!20}
\multicolumn{1}{|c|}{{\xiaowuhei 序号}} & \multicolumn{1}{c|}{{\xiaowuhei 工作}} & \multicolumn{1}{c|}{{\xiaowuhei 职责}} & \multicolumn{1}{c|}{{\xiaowuhei 负责人}} \\
\hline
\multicolumn{1}{|c|}{{\xiaowu 1}} & {\xiaowu 获取和开发测试环境元素} & {\xiaowu 逐项获取或开发软件测试环境涉及到的软件项、硬件和固件项、其他项、其他材料等} & {\xiaowu 李胜奎} \\
\hline
\multicolumn{1}{|c|}{{\xiaowu 2}} & {\xiaowu 建立软件测试环境} & {\xiaowu 测试开始前,安装调试软件测试环境,使其达到测试要求} & {\xiaowu 杭滔} \\
\hline
\multicolumn{1}{|c|}{{\xiaowu 3}} & {\xiaowu 控制和维护软件测试环境} & {\xiaowu 在测试过程中,控制和维护软件测试环境中的每个项} & {\xiaowu 李胜奎} \\
\hline
\end{tabular}
}
\end{table}
\vspace{-6pt}  % 表格后为标题,总间距6pt

}

\subsubsection*{3.2.6 测试环境的差异性分析和有效性说明}

{\normalsize
xxxxxxxxxx测试环境满足合同指标要求中对计算机资源的要求,对测试结果不产生影响。
}

\subsubsection*{3.2.7 参与组织}

{\normalsize
本次测试的参与组织如表7所示。

% 表格使用[H]参数强制在当前位置,防止标题与内容分离
\begin{table}[H]
\centering
\vspace{6pt}
{\wuhaohei 表 7 参与组织}

\vspace{6pt}
{\settablespacing
\begin{tabular}{|p{1cm}|p{3.5cm}|p{6.5cm}|p{3.5cm}|}
\hline
\rowcolor{gray!20}
\multicolumn{1}{|c|}{{\xiaowuhei 序号}} & \multicolumn{1}{c|}{{\xiaowuhei 参与组织}} & \multicolumn{1}{c|}{{\xiaowuhei 职责}} & \multicolumn{1}{c|}{{\xiaowuhei 负责人}} \\
\hline
\multicolumn{1}{|c|}{{\xiaowu 1}} & {\xiaowu 测试组} & {\xiaowu 策划测试活动、测试用例设计及实现、组织实施测试、测试总结、出具测试报告} & {\xiaowu 李胜奎} \\
\hline
\multicolumn{1}{|c|}{{\xiaowu 2}} & {\xiaowu 开发组} & {\xiaowu 参与测试需求分析,明确测试的范围和边界,解决测试过程中发现的bug} & {\xiaowu 陈幻} \\
\hline
\multicolumn{1}{|c|}{{\xiaowu 3}} & {\xiaowu 质量管理组} & {\xiaowu 测试过程中的质量管理相关工作} & {\xiaowu 任静、李闺臣} \\
\hline
\multicolumn{1}{|c|}{{\xiaowu 4}} & {\xiaowu 配置管理组} & {\xiaowu 出具测试版本,测试过程中的版本控制} & {\xiaowu 杭滔} \\
\hline
\end{tabular}
}
\end{table}
\vspace{-6pt}  % 表格后为标题,总间距6pt

}

\subsubsection*{3.2.8 人员与分工}

{\normalsize
本次测试人员与分工安排如表8所示。

% 表格使用[H]参数强制在当前位置,防止标题与内容分离
\begin{table}[H]
\centering
\vspace{6pt}
{\wuhaohei 表 8 人员与分工安排}

\vspace{6pt}
{\settablespacing
\begin{tabular}{|p{0.5cm}|p{4.5cm}|p{2.5cm}|p{2.5cm}|p{4.5cm}|}
\hline
\rowcolor{gray!20}
\multicolumn{1}{|c|}{{\xiaowuhei 序号}} & \multicolumn{1}{c|}{{\xiaowuhei 测试活动名称}} & \multicolumn{1}{c|}{{\xiaowuhei 开始日期}} & \multicolumn{1}{c|}{{\xiaowuhei 结束日期}} & \multicolumn{1}{c|}{{\xiaowuhei 人员}} \\
\hline
\multicolumn{1}{|c|}{{\xiaowu 1}} & {\xiaowu 组织测试策划、测试需求分析,编写测试大纲} & {\xiaowu 2025.9.10} & {\xiaowu 2025.9.15} & {\xiaowu 李胜奎,李闺臣} \\
\hline
\multicolumn{1}{|c|}{{\xiaowu 2}} & {\xiaowu 测试设计,设计测试用例,编写测试细则、部署测试环境} & {\xiaowu 2025.9.13} & {\xiaowu 2025.9.18} & {\xiaowu 喻昕、胡栋、李胜奎、杭滔} \\
\hline
\multicolumn{1}{|c|}{{\xiaowu 3}} & {\xiaowu 开展功能测试(包含回归测试)并记录测试结果} & {\xiaowu 2025.9.18} & {\xiaowu 2025.9.25} & {\xiaowu 喻昕、胡栋} \\
\hline
\multicolumn{1}{|c|}{{\xiaowu 4}} & {\xiaowu 负责监督软件测试过程、软件问题报告单的闭环控制、评估测试活动} & {\xiaowu 2025.9.18} & {\xiaowu 2025.9.25} & {\xiaowu 周文俊,钟佳惠} \\
\hline
\multicolumn{1}{|c|}{{\xiaowu 5}} & {\xiaowu 负责软件版本管理,测试文档的维护管理} & {\xiaowu 2025.9.10} & {\xiaowu 2025.9.25} & {\xiaowu 李闺臣、李胜奎} \\
\hline
\multicolumn{1}{|c|}{{\xiaowu 6}} & {\xiaowu 测试总结,分析测试数据,总结测试工作,编写测试报告} & {\xiaowu 2025.9.20} & {\xiaowu 2025.9.30} & {\xiaowu 李胜奎、喻昕} \\
\hline
\end{tabular}
}
\end{table}
\vspace{-6pt}  % 表格后为标题,总间距6pt

}

\subsubsection*{3.2.9 人员培训}

{\normalsize
本次测试相关人员均有相关测试经验,故无需进行人员培训。
}

\subsubsection*{3.2.10 要执行的测试}

{\normalsize
要执行的测试详见4.2节。
}

}

% =================== 第4章:测试说明 ===================
\noindent\section*{4. 测试说明}

{\normalsize
\subsection*{4.1 一般信息}

{\normalsize
xxxxxxxxxxxxxxxxxxxxxxxxxxxxxxxxxxxxxxxxxxxxxxxxxxxxxxxxxxxxxxxxxxxxxxxxxxxxxx。

依据xxxxxxxxxx要求,结合被测项目技术成果的结构和复杂性,对测试任务进行需求分析,确定测试级别为部件测试。
}

\subsubsection*{4.1.1 测试级别}

{\normalsize
本次测试级别为部件测试,对xxxxxxxxxx系统软件单元和软件部件进行测试。
}

\subsubsection*{4.1.2 测试方法}

{\normalsize
本次测试采用的测试类型包括:功能测试、接口测试和可靠性测试。

功能测试、接口测试和可靠性测试均使用等价类划分、正交法、场景法等分析方法设计测试用例。其中功能测试使用手工测试的方法,测试人员通过界面进行相关功能测试用例,查看操作预期结果是否与实际相符;接口测试和可靠性测试则使用自动化测试的方法,测试人员通过运行测试脚本进行验证预期结果是否与实际相符。
}

\subsubsection*{4.1.3 测试类别}

{\normalsize
执行的测试类别为:功能测试、接口测试和可靠性测试。

测试用例应按照等价类划分、边界值分析、错误猜测、因果图、判定表等策略设计,测试依据中涉及的软件质量特性覆盖率应达到100\%。
}

\subsubsection*{4.1.4 一般测试条件}

{\normalsize
计划开展的测试按照优先级排序从高到低依次开展工作。计划的测试顺序如表9所示。

% 表格使用[H]参数强制在当前位置,防止标题与内容分离
\begin{table}[H]
\centering
\vspace{6pt}
{\wuhaohei 表 9 计划的测试顺序}

\vspace{6pt}
{\settablespacing
\begin{tabular}{|p{0.5cm}|p{4cm}|p{3cm}|p{3.5cm}|p{3.5cm}|}
\hline
\rowcolor{gray!20}
\multicolumn{1}{|c|}{{\xiaowuhei 序号}} & \multicolumn{1}{c|}{{\xiaowuhei 测试项名称}} & \multicolumn{1}{c|}{{\xiaowuhei 计划开始测试时间}} & \multicolumn{1}{c|}{{\xiaowuhei 计划完成测试时间}} & \multicolumn{1}{c|}{{\xiaowuhei 测试类别}} \\
\hline
\multicolumn{1}{|c|}{{\xiaowu 1}} & {\xiaowu xxxxxxxx功能测试} & {\xiaowu 2025.9.20} & {\xiaowu 2025.9.30} & {\xiaowu 功能测试} \\
\hline
\multicolumn{1}{|c|}{{\xiaowu 2}} & {\xiaowu xxxxxxxx接口测试} & {\xiaowu 2025.9.20} & {\xiaowu 2025.9.30} & {\xiaowu 接口测试} \\
\hline
\multicolumn{1}{|c|}{{\xiaowu 3}} & {\xiaowu xxxxxxxx可靠性测试} & {\xiaowu 2025.9.20} & {\xiaowu 2025.9.30} & {\xiaowu 可靠性测试} \\
\hline
\end{tabular}
}
\end{table}
\vspace{-6pt}  % 表格后为标题,总间距6pt

}

\subsubsection*{4.1.5 测试进展}

{\normalsize
测试进度安排详见第5章。
}

\subsubsection*{4.1.6 数据记录、整理和分析}

{\normalsize
软件测试期间获得的测试数据或测试过程中用到的数据,采用手工方式进行整理和分析,将整理和分析得到的信息保存并存档。整理和分析数据的结果包括:各类测试的测试结果、测试发现的问题和测试人员建议的改进意见等。

在xxxxxxxxxx系统软件需求规格说明中描述的PQL查询语言交互功能为"主要提供数据世系的血缘查询和可信查询,其中血缘查询可支持查询某个数字对象的来源、后继情况,以及两个数字对象之间的血缘联通情况、是否有同源祖先等情况;可信查询主要可查询基于数据性质的可信情况",因此该文档的编写按照满足数据世系的血缘查询和可信查询进行测试项的划分。

围绕xxxxxxxxxx和《xxxxxxxxxx系统软件需求规格说明》的要求计划开展的测试共计5个测试项其中功能测试2个,接口测试2个和可靠性测试1个。
}

\subsubsection*{4.1.7 与测试有关的安全性和保密性}

{\normalsize
xxxxxxxxxxxxxxxxxxxxxxxxxxxxxxxxxxxxxxxxxxxxxxxxxxxxxxxxxxxxxxxxxxxxxxxxxxxxxx。

测试过程中所有相关人员必须严格遵守保密规定,不得泄露测试内容、测试数据和测试结果。
}

\subsection*{4.2 计划执行的测试}

{\normalsize
计划执行的测试如表10所示。

% 表格使用[H]参数强制在当前位置,防止标题与内容分离
\begin{table}[H]
\centering
\vspace{6pt}
{\wuhaohei 表 10 测试项列表}

\vspace{6pt}
{\settablespacing
\begin{tabular}{|p{1cm}|p{4cm}|p{3cm}|p{6.5cm}|}
\hline
\rowcolor{gray!20}
\multicolumn{1}{|c|}{{\xiaowuhei 序号}} & \multicolumn{1}{c|}{{\xiaowuhei 测试类别}} & \multicolumn{1}{c|}{{\xiaowuhei 测试项名称}} & \multicolumn{1}{c|}{{\xiaowuhei 说明}} \\
\hline
\multicolumn{1}{|c|}{{\xiaowu 1}} & {\xiaowu 功能测试} & {\xiaowu xxxxxxxx功能测试} & {\xiaowu 基于PQL查询语言的数据血缘和可信查询功能测试} \\
\hline
\multicolumn{1}{|c|}{{\xiaowu 2}} & {\xiaowu 功能测试} & {\xiaowu xxxxxxxx功能测试} & {\xiaowu 基于PQL查询语言的数据血缘和可信查询功能测试} \\
\hline
\multicolumn{1}{|c|}{{\xiaowu 3}} & {\xiaowu 接口测试} & {\xiaowu xxxxxxxx接口测试} & {\xiaowu 测试各模块之间的接口参数传递正确性} \\
\hline
\multicolumn{1}{|c|}{{\xiaowu 4}} & {\xiaowu 接口测试} & {\xiaowu xxxxxxxx接口测试} & {\xiaowu 测试各模块之间的接口参数传递正确性} \\
\hline
\multicolumn{1}{|c|}{{\xiaowu 5}} & {\xiaowu 可靠性测试} & {\xiaowu xxxxxxxx可靠性测试} & {\xiaowu 测试系统在异常情况下的恢复能力} \\
\hline
\end{tabular}
}
\end{table}
\vspace{-6pt}  % 表格后为标题,总间距6pt

}

\subsubsection*{4.2.1 xxxxx语言功能测试(xxxxxx\_GN)}

{\normalsize
本测试项的测试要求、策略与方法如下:

测试要求:依据《xxxxxxxxxx系统软件需求规格说明》,对基于PQL查询语言的血缘演算功能进行测试。

测试策略:
测试方法:该测试使用手工测试的方法,测试人员通过界面执行相关功能测试用例,查看操作预期结果是否与实际相符。

测试终止条件:全部测试用例被执行或因某种原因导致测试无法执行。

测试通过准则:软件测试问题报告中1级、2级问题均已关闭,且3级、4级问题关闭率达到80\%以上,则测试通过,否则测试不通过。

\paragraph*{4.2.1.1 xxxxx功能测试(xxxxxx\_GN)}

{\normalsize
测试进度如表11所示。

% 表格使用[H]参数强制在当前位置,防止标题与内容分离
\begin{table}[H]
\centering
\vspace{6pt}
{\wuhaohei 表 11 基于PQL查询语言的血缘演算功能测试}

\vspace{6pt}
{\settablespacing
\begin{tabular}{|p{3cm}|p{6.5cm}|p{5cm}|}
\hline
\rowcolor{gray!20}
\multicolumn{1}{|c|}{{\xiaowuhei 测试项名称}} & \multicolumn{1}{c|}{{\xiaowuhei 测试要求}} & \multicolumn{1}{c|}{{\xiaowuhei 测试策略与方法}} \\
\hline
\multicolumn{1}{|c|}{{\xiaowu xxxxxxxx功能测试}} & {\xiaowu 依据《xxxxxxxxxx系统软件需求规格说明》,对基于PQL查询语言的血缘演算功能进行测试。} & {\xiaowu 测试策略:} \\
\hline
\end{tabular}
}
\end{table}
\vspace{6pt}  % 表格后为正文,总间距18pt

测试方法:该测试使用手工测试的方法,测试人员通过界面执行相关功能测试用例,查看操作预期结果是否与实际相符。

测试终止条件:全部测试用例被执行或因某种原因导致测试无法执行。

测试通过准则:软件测试问题报告中1级、2级问题均已关闭,且3级、4级问题关闭率达到80\%以上,则测试通过,否则测试不通过。

}

% =================== 第5章:测试进度 ===================
\noindent\section*{5. 测试进度}

{\normalsize
本次测试进度安排如表12所示。

% 表格使用[H]参数强制在当前位置,防止标题与内容分离
\begin{table}[H]
\centering
\vspace{6pt}
{\wuhaohei 表 12 测试进度表}

\vspace{6pt}
{\settablespacing
\begin{tabular}{|p{2.5cm}|p{3.5cm}|p{3.5cm}|p{5cm}|}
\hline
\rowcolor{gray!20}
\multicolumn{1}{|c|}{{\xiaowuhei 测试活动}} & \multicolumn{1}{c|}{{\xiaowuhei 起止时间}} & \multicolumn{1}{c|}{{\xiaowuhei 内容}} & \multicolumn{1}{c|}{{\xiaowuhei 输出}} \\
\hline
\multicolumn{1}{|c|}{{\xiaowu 测试策划}} & {\xiaowu 2025/09/10-2025/09/15} & {\xiaowu 完成测试前期的测试需求分析,制定测试大纲、测试项} & {\xiaowu 测试大纲初稿} \\
\hline
\multicolumn{1}{|c|}{{\xiaowu 测试设计与实现}} & {\xiaowu 2025/09/13-2025/09/18} & {\xiaowu 搭建测试环境、设计测试用例,准备测试数据} & {\xiaowu 测试细则初稿} \\
\hline
\multicolumn{1}{|c|}{{\xiaowu 测试评审}} & {\xiaowu 2025/09/29} & {\xiaowu 测试大纲、测试细则评审} & {\xiaowu 测试大纲、测试细则、评审结论} \\
\hline
\multicolumn{1}{|c|}{{\xiaowu 实施测试}} & {\xiaowu 2025/09/30} & {\xiaowu 按测试细则要求执行测试用例,记录测试结果和测试问题} & {\xiaowu 测试记录、测试问题报告} \\
\hline
\multicolumn{1}{|c|}{{\xiaowu 回归测试}} & {\xiaowu 2025/09/30} & {\xiaowu 对修复的问题进行回归测试} & {\xiaowu 回归测试记录} \\
\hline
\multicolumn{1}{|c|}{{\xiaowu 测试总结}} & {\xiaowu 2025/09/30} & {\xiaowu 根据测试结果和测试记录,撰写软件测评报告} & {\xiaowu 测试报告、测试结论} \\
\hline
\end{tabular}
}
\end{table}
\vspace{-6pt}  % 表格后为标题,总间距6pt

}

% =================== 第6章:测试终止条件 ===================
\noindent\section*{6. 测试终止条件}

{\normalsize
本次测试的终止条件如表13所示。

% 表格使用[H]参数强制在当前位置,防止标题与内容分离
\begin{table}[H]
\centering
\vspace{6pt}
{\wuhaohei 表 13 缺陷级别}

\vspace{6pt}
{\settablespacing
\begin{tabular}{|p{1.5cm}|p{3cm}|p{10cm}|}
\hline
\rowcolor{gray!20}
\multicolumn{1}{|c|}{{\xiaowuhei 缺陷级别}} & \multicolumn{1}{c|}{{\xiaowuhei 描述}} & \multicolumn{1}{c|}{{\xiaowuhei 说明}} \\
\hline
\multicolumn{1}{|c|}{{\xiaowu 1级}} & {\xiaowu 致命问题} & {\xiaowu 程序运行过程中不断申请但没有完全释放资源,造成系统性能越来越低并出现无规律的死机现象。程序运行导致系统崩溃。由程序引起的资源严重不足、非法退出等。严重的关键计算错误(如计费等)。数据库发生死锁且无法自动恢复。与需求要求差距较大。系统无响应} \\
\hline
\multicolumn{1}{|c|}{{\xiaowu 2级}} & {\xiaowu 严重问题} & {\xiaowu 因错误操作导致的程序中断。功能没有实现。正确操作导致的错误结果。与数据库连接错误且无法自动恢复。数据通讯错误且无法自动恢复。数据库的表、业务规则、缺省值未加完整性等约束条件。界面中的信息不能及时刷新不能正确反映当前数据状态可能误导用户(数据库中剩余记录个数和参数设置对话框中的预设值常常显示为历史值而不是当前值)。对输入数据没有进行充分并且有效的有效性检查造成不合要求的数据进入数据库} \\
\hline
\multicolumn{1}{|c|}{{\xiaowu 3级}} & {\xiaowu 一般问题} & {\xiaowu 操作界面错误(包括数据窗口内列名定义、含义是否一致界面中英文混杂界面元素参差不齐文字显示不全)。打印内容、格式错误。删除操作未给出提示。数据库表中有过多的空字段。提示信息意文不明或为原始的英文提示。要求用户输入多余的、本来系统可以自动获取的数据(服务是否启动安装后用户需要手动修改某些配置文件)。辅助说明描述不清楚有歧义。长操作未给用户提示} \\
\hline
\multicolumn{1}{|c|}{{\xiaowu 4级}} & {\xiaowu 改进建议} & {\xiaowu 辅助说明描述不清楚。输入输出不规范。可输入区域和只读区域没有明显的区分标志。某一项功能的冗余操作太多如对话框嵌套层次太多影响用户操作。不能记忆用户的设置或操作习惯用户每次进入系统都需要重新操作初始环境。不符合用户操作习惯(快捷键定义不科学不实用键位分布不合理按键太多甚至没有快捷键)。界面不规范。提示窗口文字未采用行业术语} \\
\hline
\end{tabular}
}
\end{table}
\vspace{6pt}  % 表格后为正文,总间距18pt

测试终止条件为:
(1)全部测试用例被执行完毕;
(2)测试问题报告中1级、2级问题全部关闭;
(3)测试问题报告中3级、4级问题关闭率达到80\%以上。
}

% =================== 第7章:需求的可追踪性 ===================
\noindent\section*{7. 需求的可追踪性}

{\normalsize
本测试大纲对《xxxxxxxxxx系统软件需求规格说明》的可追踪性如表14和表15所示。

% 表格使用[H]参数强制在当前位置,防止标题与内容分离
\begin{table}[H]
\centering
\vspace{6pt}
{\wuhaohei 表 14 xxxxxxxxxx与需求规格说明以及测试项的追踪关系表}

\vspace{6pt}
{\settablespacing
\begin{tabular}{|p{1cm}|p{3cm}|p{3cm}|p{7cm}|}
\hline
\rowcolor{gray!20}
\multicolumn{1}{|c|}{{\xiaowuhei 序号}} & \multicolumn{1}{c|}{{\xiaowuhei 需求名称/标识}} & \multicolumn{1}{c|}{{\xiaowuhei 需求规格说明章节号}} & \multicolumn{1}{c|}{{\xiaowuhei 测试项名称/标识}} \\
\hline
\multicolumn{1}{|c|}{{\xiaowu 1}} & {\xiaowu 数据血缘查询功能} & {\xiaowu 4.x} & {\xiaowu xxxxxxxx功能测试} \\
\hline
\multicolumn{1}{|c|}{{\xiaowu 2}} & {\xiaowu 数据可信查询功能} & {\xiaowu 4.x} & {\xiaowu xxxxxxxx功能测试} \\
\hline
\multicolumn{1}{|c|}{{\xiaowu 3}} & {\xiaowu 模块间接口} & {\xiaowu 4.x} & {\xiaowu xxxxxxxx接口测试} \\
\hline
\multicolumn{1}{|c|}{{\xiaowu 4}} & {\xiaowu 系统可靠性} & {\xiaowu 4.x} & {\xiaowu xxxxxxxx可靠性测试} \\
\hline
\end{tabular}
}
\end{table}
\vspace{6pt}  % 表格后为正文,总间距18pt

xxxxxxxxxx系统软件需求规格说明中描述的PQL查询语言交互功能,本测试大纲的第4.2节对此进行了覆盖。

% 表格使用[H]参数强制在当前位置,防止标题与内容分离
\begin{table}[H]
\centering
\vspace{6pt}
{\wuhaohei 表 15 xxxxxxxxxx与需求规格说明以及测试项的逆向追踪关系表}

\vspace{6pt}
{\settablespacing
\begin{tabular}{|p{1cm}|p{3cm}|p{3cm}|p{7cm}|}
\hline
\rowcolor{gray!20}
\multicolumn{1}{|c|}{{\xiaowuhei 序号}} & \multicolumn{1}{c|}{{\xiaowuhei 测试项名称/标识}} & \multicolumn{1}{c|}{{\xiaowuhei 本文档的章节号}} & \multicolumn{1}{c|}{{\xiaowuhei 需求名称/标识}} \\
\hline
\multicolumn{1}{|c|}{{\xiaowu 1}} & {\xiaowu xxxxxxxx功能测试} & {\xiaowu 4.1、4.2} & {\xiaowu 数据血缘查询功能、数据可信查询功能} \\
\hline
\multicolumn{1}{|c|}{{\xiaowu 2}} & {\xiaowu xxxxxxxx接口测试} & {\xiaowu 4.2} & {\xiaowu 模块间接口} \\
\hline
\multicolumn{1}{|c|}{{\xiaowu 3}} & {\xiaowu xxxxxxxx可靠性测试} & {\xiaowu 4.2} & {\xiaowu 系统可靠性} \\
\hline
\end{tabular}
}
\end{table}
\vspace{-6pt}  % 表格后为标题,总间距6pt

}

% =================== 第8章:注释 ===================
\noindent\section*{8. 注释}

{\normalsize
本文档适用于xxxxxxxxxx系统项目2025年9月节点测试。

本测试大纲依据GJB438C-2021《军用软件开发文档通用要求》编制,用于指导本次测试活动。

本文档密级为xx密,必须按照机密文件的要求使用和分发。
}

\end{document}