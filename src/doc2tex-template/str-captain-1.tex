% 技术文件 - 严格按照格式标准
\documentclass[12pt,a4paper]{article}

% 使用标准包
\usepackage{ctex}
\usepackage{fontspec}  % 字体设置
\usepackage{xcolor}  % 用于行颜色
\usepackage{colortbl}  % 表格颜色支持
\usepackage{geometry}  % 页面设置
\usepackage{setspace}  % 行间距设置
\usepackage{titlesec}  % 标题格式设置

% 按标准设置中文字体
\setCJKmainfont{Noto Serif CJK SC}  % 宋体
\setCJKsansfont{Noto Sans CJK SC}  % 黑体
\setmainfont{DejaVu Serif}  % 西文字体

% 按标准定义字号命令
\newcommand{\wuhao}{\fontsize{10.5pt}{15.75pt}\selectfont}  % 五号
\newcommand{\wuhaohei}{\wuhao\bfseries}  % 五号黑体加粗
\newcommand{\xiaowu}{\fontsize{9pt}{11pt}\selectfont}  % 小五号
\newcommand{\xiaowusong}{\xiaowu\rmfamily}  % 小五号宋体
\newcommand{\xiaowuhei}{\xiaowu\bfseries}  % 小五号黑体加粗
\newcommand{\sanhao}{\fontsize{16pt}{18pt}\selectfont\bfseries}  % 三号黑体加粗

% 按标准定义正文字体为五号宋体,西文为Times New Roman
\renewcommand{\normalsize}{\wuhao}

% 页面设置
\geometry{a4paper,left=2.5cm,right=2.5cm,top=2.5cm,bottom=2.5cm}

% 设置段落首行缩进为标准中文缩进
\setlength{\parindent}{1.5em}
\setlength{\parskip}{0pt}

% 设置行间距为固定值18磅 (18/10.5 ≈ 1.71)
\renewcommand{\baselinestretch}{1.71}

% 设置页码为小五号新罗马体,版心下面居中
\pagestyle{plain}
\renewcommand{\thepage}{\xiaowu\rmfamily\arabic{page}}

% 表格标题格式 - 五号黑体
\renewcommand{\tablename}{\wuhaohei 表}
\renewcommand{\figurename}{\wuhaohei 图}

% === 表格行间距设置 ===
% 定义表格行间距倍数(设置为1.0,使用默认表格行间距)
\newcommand{\tablearraystretch}{1.0}
% 应用表格行间距的辅助命令
\newcommand{\settablespacing}{\renewcommand{\arraystretch}{\tablearraystretch}}
\newcommand{\resettablespacing}{\renewcommand{\arraystretch}{1.0}}

% 设置1级、2级、3级标题格式:五号宋体加粗,标题之间间距6磅,行间距18磅
\titleformat{\section}
  {\wuhao\bfseries}
  {\thesection}
  {1em}
  {}
\titlespacing*{\section}{0pt}{6pt}{0pt}  % 段前6pt,段后0pt,标题之间间距为6pt

\titleformat{\subsection}
  {\wuhao\bfseries}
  {\thesubsection}
  {1em}
  {}
\titlespacing*{\subsection}{0pt}{6pt}{0pt}  % 段前6pt,段后0pt,标题之间间距为6pt

\titleformat{\subsubsection}
  {\wuhao\bfseries}
  {\thesubsubsection}
  {1em}
  {}
\titlespacing*{\subsubsection}{0pt}{6pt}{0pt}  % 段前6pt,段后0pt,标题之间间距为6pt

\begin{document}

% 移除封面和目录,直接从第1章开始

% =================== 第一章:范围 ===================
\noindent\section*{1. 范围}

{\normalsize
\subsection*{1.1 标识}

{\normalsize
(1)文档标识号:xxxx;

(2)系统标识:xxxx,测试文档简称为xxxx;

(3)项目名称:xxxxxx;

(4)文档名称:xxxxxx节点软件测试大纲;

(5)软件版本号:xxxx版,软件版本标识采用三位编码规则,项目文档版本标识采用四位编码规则;

(6)本文档适用于:xxxxx系统项目。
}

\subsection*{1.2 系统概述}

{\normalsize
系统用途部分内容敏感,不在此处列出。详见xxxxxx系统合同(xxxxxx)。

xxxxxxx规定的本阶段主要要求和技术指标为:

xxxxxxxxxxxxxxxxxxxxxxxxxxxx。

项目节点测试的测试项与xxxxxxx的覆盖性对应关系如表1所示。

\vspace{-12pt}
\begin{center}
{\wuhaohei 表 1 xxxxxx与测试项覆盖性对照表}

{\settablespacing
\begin{tabular}{|p{0.5cm}|p{7cm}|p{7cm}|}
\hline
\rowcolor{gray!20}
\multicolumn{1}{|c|}{{\xiaowuhei 序号}} & \multicolumn{1}{c|}{{\xiaowuhei xxxxxx}} & \multicolumn{1}{c|}{{\xiaowuhei 测试项}} \\
\hline
\multicolumn{1}{|c|}{{\xiaowu 1}} & {\xiaowu } & {\xiaowu } \\
\hline
\multicolumn{1}{|c|}{{\xiaowu 2}} & {\xiaowu } & {\xiaowu } \\
\hline
\multicolumn{1}{|c|}{{\xiaowu 3}} & {\xiaowu } & {\xiaowu } \\
\hline
\multicolumn{1}{|c|}{{\xiaowu 4}} & {\xiaowu } & {\xiaowu } \\
\hline
\multicolumn{1}{|c|}{{\xiaowu 5}} & {\xiaowu } & {\xiaowu } \\
\hline
\end{tabular}
}
\end{center}
\vspace{6pt}

xxxxxxxxxx系统软件的需方是"M"项目管理办公室,开发方是xxxxx。

标识当前和计划运行的现场,测试地点为"xxxxxx",测试环境包括1台国产ARM架构服务器,其搭载国产处理器(飞腾)和国产操作系统(银河麒麟服务器版)以及2台国产X86台式机,其中服务器部署xxxxxxxxxxreleaseV0.7.1版,两台台式机为客户端。

\subsection*{1.3 文档概述}

{\normalsize
本文档用于"xxxxxxxxxx系统"项目2025年9月节点测试大纲,规定了本次测试的类型、内容、环境、方法、条件及各项测试活动的进度安排,指导测试人员进行测试活动,为编写软件测试细则以及测试报告提供依据及为后续测试工作打下基础。

本文档模板涵盖了GJB438C-2021《军用软件开发文档通用要求》对软件测试大纲文档的要素和内容的要求。

描述本文档密级为xx密,必须按照机密文件的要求使用和分发。即使在保密性得到保证的情况下,对本文档的使用也必须经研制方主管人员的认可。
}

\subsection*{1.4 与其他计划的关系}

{\normalsize
本文档属于"xxxxxxxxxx系统"项目2025年9月节点测试的一部分,测试人员根据xxxxxxxxxx系统项目《软件需求规格说明》识别被测软件的功能和结构,确定测试环境,收集测试需求以标识被测试项,安排测试进度以及进行测试活动。
}

\subsection*{1.5 引用文档}

\vspace{-12pt}
\begin{center}
{\wuhaohei 表 2 引用文档一览表}

{\settablespacing
\begin{tabular}{|p{0.5cm}|p{2.5cm}|p{5.5cm}|p{2.5cm}|p{1.5cm}|p{2cm}|}
\hline
\rowcolor{gray!20}
\multicolumn{1}{|c|}{{\xiaowuhei 序号}} & \multicolumn{1}{c|}{{\xiaowuhei 编号}} & \multicolumn{1}{c|}{{\xiaowuhei 标题}} & \multicolumn{1}{c|}{{\xiaowuhei 编写单位}} & \multicolumn{1}{c|}{{\xiaowuhei 日期}} & \multicolumn{1}{c|}{{\xiaowuhei 备注}} \\
\hline
\multicolumn{1}{|c|}{{\xiaowu 1}} & {\xiaowu GJB 438C-2021} & {\xiaowu 军用软件开发文档通用要求} & {\xiaowu xxxxx} & {\xiaowu 2021} & {\xiaowu /} \\
\hline
\multicolumn{1}{|c|}{{\xiaowu 2}} & {\xiaowu /} & {\xiaowu xxxxxxxxxx} & {\xiaowu xxxxx} & {\xiaowu 2025} & {\xiaowu /} \\
\hline
\multicolumn{1}{|c|}{{\xiaowu 3}} & {\xiaowu xxxxxxxxxx–SRS} & {\xiaowu xxxxxxxxxx系统软件需求规格说明} & {\xiaowu xxxxx} & {\xiaowu 2025} & {\xiaowu /} \\
\hline
\end{tabular}
}
\end{center}
\vspace{6pt}

\end{document}