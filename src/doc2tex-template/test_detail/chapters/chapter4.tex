\section*{4. 测试说明}

{\normalsize
\subsection*{4.1 计划执行的测试}

{\normalsize
计划执行的测试如表4所示。

\begin{table}[H]
\centering
\vspace{6pt}
{\wuhaohei 表 4 测试项列表}

\vspace{6pt}
{\settablespacing
\begin{tabular}{|p{1cm}|p{4cm}|p{3cm}|p{6.5cm}|}
\hline
\rowcolor{gray!20}
\multicolumn{1}{|c|}{{\xiaowuhei 序号}} & \multicolumn{1}{c|}{{\xiaowuhei 测试类别}} & \multicolumn{1}{c|}{{\xiaowuhei 测试项名称}} & \multicolumn{1}{c|}{{\xiaowuhei 说明}} \\
\hline
\multicolumn{1}{|c|}{{\xiaowu 1}} & {\xiaowu 功能测试} & {\xiaowu xxxxxxxx功能测试} & {\xiaowu 基于PQL查询语言的数据血缘和可信查询功能测试} \\
\hline
\multicolumn{1}{|c|}{{\xiaowu 2}} & {\xiaowu 功能测试} & {\xiaowu xxxxxxxx功能测试} & {\xiaowu 基于PQL查询语言的数据血缘和可信查询功能测试} \\
\hline
\multicolumn{1}{|c|}{{\xiaowu 3}} & {\xiaowu 接口测试} & {\xiaowu xxxxxxxx接口测试} & {\xiaowu 测试各模块之间的接口参数传递正确性} \\
\hline
\multicolumn{1}{|c|}{{\xiaowu 4}} & {\xiaowu 接口测试} & {\xiaowu xxxxxxxx接口测试} & {\xiaowu 测试各模块之间的接口参数传递正确性} \\
\hline
\multicolumn{1}{|c|}{{\xiaowu 5}} & {\xiaowu 可靠性测试} & {\xiaowu xxxxxxxx可靠性测试} & {\xiaowu 测试系统在异常情况下的恢复能力} \\
\hline
\end{tabular}
}
\end{table}
\vspace{-6pt}

}

\subsubsection*{4.1.1 面向xxxxx的xxxxx功能测试(xxxx\_xxxxx\_GN)}

{\normalsize
本测试项的测试要求、策略与方法如下:

测试要求:依据《xxxxxxxxxx系统软件需求规格说明》,对基于PQL查询语言的血缘演算功能进行测试。

测试策略:xxxxxxxxxxxxxxxxxxxxxxxxxxxxxxxxxxxx。

测试方法:该测试使用手工测试的方法,测试人员通过界面执行相关功能测试用例,查看操作预期结果是否与实际相符。

测试终止条件:全部测试用例被执行或因某种原因导致测试无法执行。

测试通过准则:软件测试问题报告中1级、2级问题均已关闭,且3级、4级问题关闭率达到80\%以上,则测试通过,否则测试不通过。

\paragraph*{4.1.1.1 基于xxxxx的xxxxx测试(xxxx\_xxxxx)}

{\normalsize
\subparagraph*{4.1.1.1.1 xxxxx功能测试(xxxx\_xxxxx\_001)}

{\normalsize
用例ID:xxxx\_xxxxx\_001

测试目标:xxxxxxxxxxxxxxxxxxxxxxxxxxxxxxxxxxxx

前置条件:
(1)xxxxxxxxxxxxxxxxxxxxxxxxxxxxxxxxxxxx;
(2)xxxxxxxxxxxxxxxxxxxxxxxxxxxxxxxxxxxx。

测试步骤:
(1)xxxxxxxxxxxxxxxxxxxxxxxxxxxxxxxxxxxx;
(2)xxxxxxxxxxxxxxxxxxxxxxxxxxxxxxxxxxxx;
(3)xxxxxxxxxxxxxxxxxxxxxxxxxxxxxxxxxxxx。

预期结果:xxxxxxxxxxxxxxxxxxxxxxxxxxxxxxxxxxxx

备注:xxxxxxxxxxxxxxxxxxxxxxxxxxxxxxxxxxxx
}

}

}

% =================== 第5章:需求的可追踪性 ===================
\noindent